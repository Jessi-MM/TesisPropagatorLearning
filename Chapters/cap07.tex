%************************************************
\chapter{Conclusiones}\label{ch:Conclusiones}
% ************************************************
La evolución temporal de un sistema cuántico está descrito por la ecuación de Schrödinger \autoref{eq:TDSE ket}. Esta ecuación puede ser resuelta al encontrar el operador de propagación que evoluciona en el tiempo a la función de oda del sistema, lo cual es, en general, un proceso complicado para los métodos analíticos, o costoso en tiempo computacional para los métodos numéricos. En este trabajo de tesis se desarrolló una alternativa para resolver la \acs{TDSE} utilizando una red neuronal artificial recurrente tipo long short‐term memory (\acs{LSTM}) como propagador, encontrando que la red entrenada necesita menos tiempo de ejecución ($\approx 70\%$ menos tiempo) para propagar una función de onda en comparación con el método de referencia \acs{DVR}. En este trabajo se logró una precisión promedio de $0.946$ para la magnitud $|S|$ y $0.0001$ para la fase absoluta $\theta$. Con esto se demostró que una red neuronal recurrente es un modelo efectivo para construir un propagador cuántico para problemas unidimensionales.\\

Como propuesta para trabajos futuros se presentan las siguientes sugerencias:
\begin{itemize}[label=\textcolor{CTtitle}{\textbullet}]
\item Utilizar más puntos en la malla: Generar datos con más puntos en la malla requiere de un mayor tiempo de computación, sin embargo la calidad en las predicciones puede mejorar significativamente, esperando resultados como los que se presentan en la sección \autoref{sec:ProtonTransfer}. 
\item Implementar el modelo para dos y tres dimensiones en el espacio de posiciones: En el documento de información adicional de la referencia \cite{Main:2021} se proporcionan las ecuaciones para modelar el potencial en dos dimensiones. De hecho, es al crecer la dimensionalidad del problema cuando las verdaderas ventajas de usar redes neuronales será más obvio. Esto debido a que resolver las ecuaciones de \acs{DVR} implica un incremento exponencial con la dimensión de la función de onda, es decir, la complejidad de calculo crece como $N^{s}$, donde $N$ es la dimensión y $s$ es el número de puntos en la malla, mientras que el tiempo de ejecución de una \acs{LSTM} se mantiene constante o a lo más, crece linealmente, debido a que la complejidad de la red no varia mucho al aumentar la dimensión del problema a aprender.
\end{itemize}
Se espera que este trabajo de tesis sirva como base para atacar problemas más ambiciosos de dinámica cuántica y que siente las bases para futuras aplicaciones más directas y de interés general en la comunidad de química cuántica y ciencia de materiales.