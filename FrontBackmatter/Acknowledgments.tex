%*******************************************************
% Acknowledgments
%*******************************************************
\pdfbookmark[1]{Acknowledgments}{acknowledgments}

\bigskip

\begingroup
\let\clearpage\relax
\let\cleardoublepage\relax
\let\cleardoublepage\relax
\chapter*{Agradecimientos}

A mi asesor, el Dr. Huziel E. Sauceda, por apoyarme y ofrecerme siempre su mentoría durante el desarrollo de este proyecto, sus consejos e ideas fueron esenciales para la culminación de este trabajo.
\\
A mis sinodales, por tomarse el tiempo de leer y evaluar mi trabajo.
\\

A mis buenos profesores y compañeres, que fomentaron un entorno humanístico tan importante como el académico. En particular al Dr. José Rubén Alfaro, José Serna, Valeria, Sergio y Daniel, quienes fueron un equipo importante por brindarme ratos de mucha diversión, y apoyo en momentos difíciles.\\
A Edson por ser mi gran camarada de clases.
\\
A mis mejores amigos Diana y Carlos, por siempre estar, por sus consejos y compañía.
\\

Mis años como estudiante universitaria representaron un verdadero reto, y estaré eternamente agradecida con mi familia por la comprensión y el apoyo que siempre me brindaron. A mis padres, Irma y Antonio, sin duda todos mis pequeños y grandes logros son gracias a su guía, esfuerzo y al infinito amor con el que me criaron. A mis hermanas, Nelly y Aylen, porque siempre dieron más de ellas en el hogar para que yo pudiera dedicarme únicamente a estudiar, su compañía y alegría fueron mi refugio en los momentos más difíciles y me dieron la fuerza para no rendirme. A mis abuelitos, tías, tíos y primas, porque la universidad llegó a consumirme mucho tiempo que no pude dedicarle a ellos, ni estar en muchos momentos importantes, y aún así, el apoyo, la calidez y amor cuando nos reunimos están intactos.\\
A Kotomi y Carmina, mis lomitos, por existir.
\\
Por último, quiero agradecer al compañero de mi clase de mecánica que un día me pasó apuntes, Aldo, porque desde el día que lo conocí, su pasión por la Física me inspiró y motivó a seguir este camino, y porque cuando el camino se volvía oscuro siempre se convertía en luz; por siempre acompañarme y darme seguridad, por convertirse en el mejor amigo, el mejor novio y el mejor esposo que la vida me pudo dar.
\endgroup
