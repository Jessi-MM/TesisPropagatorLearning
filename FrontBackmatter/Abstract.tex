%*******************************************************
% Abstract
%*******************************************************
%\renewcommand{\abstractname}{Abstract}
\pdfbookmark[1]{Abstract}{Abstract}
% \addcontentsline{toc}{chapter}{\tocEntry{Abstract}}
\begingroup
\let\clearpage\relax
\let\cleardoublepage\relax
\let\cleardoublepage\relax

\chapter*{Resumen}
\small
La ecuación de Schrödinger describe la evolución temporal de sistemas cuánticos, cuando estos sistemas involucran interacciones físicas y químicas que dependen del tiempo la resolución de la ecuación puede ser complicada y compleja analíticamente o costosa numéricamente. Las redes neuronales artificiales (\acs{ANN}s) son un método de aprendizaje automático que ha ganado popularidad en los últimos años debido a los avances en computación y a su versatilidad en aplicaciones a diversas áreas, incluidas las ciencias. En este trabajo se presenta una alternativa a los métodos convencionales para resolver la ecuación de Schrödinger dependiente del tiempo y encontrar la evolución temporal en sistemas de transferencia de protones, mediante el uso de una \acs{ANN} de tipo long short-term memory. Se propagó un paquete de onda inicial bajo un potencial dependiente del tiempo en pasos de tiempo cortos durante un periodo largo de tiempo, considerando la escala de tiempo característico que tienen este tipo de procesos químicos y biológicos. Se encontró que la red entrenada puede propagar paquetes de onda utilizando un tiempo de ejecución menor al de otros métodos numéricos, proponiendo así una alternativa viable para la resolución de estos sistemas.
%\newpage

\begin{otherlanguage}{english}
%\pdfbookmark[1]{Zusammenfassung}{Zusammenfassung}
  \chapter*{Abstract}
  \small
  The Schrödinger equation describes the time evolution of quantum systems. When these systems involve time-dependent physical and chemical interactions, solving the equation can become more complicated and complex for analytical methods or computationally expensive for numerical methods. Artificial neural networks (\acs{ANN}s) are a machine learning method that has gained popularity recently due to technological advances and its versatility in applications to various areas, including the sciences. In this work, we present an alternative to conventional methods for solving the time-dependent Schrödinger equation and finding the time evolution in proton transfer systems by using a long short-term memory type \acs{ANN}, which propagates an initial wave packet under a time-dependent potential in short time steps over a long period, considering the characteristic time scale of this chemical and biological processes. We found that the trained network can propagate wave packets using a shorter execution time than other numerical methods, thus proposing a viable alternative for solving these systems.


\end{otherlanguage}

\endgroup

\vfill
